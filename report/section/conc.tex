In this paper, we have explored the virtualization detection methods mentioned
in nEther~\cite{nether} and its countermeasures. Software-based virtualization
are infeasible to be transparent due to their countless bugs. However, we have
proposed techniques for defending anomaly based \mvf{dont used the word
  ``anomaly''} techniques, such as CPU errata. Although a thorough investigation
into evaluating the effectiveness of the proposed techniques was not conducted,
we believe that anomaly based detection \mvf{dont used the word ``anomaly''}
schemes are feasible to be patched. Detecting virtualization can also have
benign use \mvf{remove or mention in a discussion section}. Virtual Machine
Based Root-kits (VMBR) make AV software have to detect virtualization as
well~\cite{thompson, ferrie2007}. Examples are BluePill~\cite{bluepill} and
Vitriol~\cite{vitriol}, both using virtualization extensions (both Intel and
AMD). Therefore, the research community should be aware of the fact that
completly transparent VMs is a double-edged sword.~\mvf{this need to be
  discusses before being included in the conclusion!}.

%%% Local Variables:
%%% mode: latex
%%% TeX-master: "../paper"
%%% End:
