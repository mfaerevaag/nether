Garfinkel {\em et al.}~\cite{garfinkel2007} believe that complete transparency
of VMM is infeasible and is contrary to the fundamental limitations of
virtualization technology. Regardless of the detection type, making the software
based virtualization transparent is infeasible due to countless bugs. On the
hardware-based type however, CPUID and CPU errata based methods are small enough to play the cat and mouse game where malware
authors find vulnerabilities and security professionals patch the vulnerable
instructions. For the timing based attacks, it is very difficult to prevent
relative comparison attacks and cache based attacks.
Detecting virtualization can also have benign use. Virtual Machine
Based Root-kits (VMBR) make AV software have to detect virtualization as
well~\cite{thompson, ferrie2007}. Examples are BluePill~\cite{bluepill} and
Vitriol~\cite{vitriol}, both using virtualization extensions (both Intel and
AMD). Therefore, the research community should be aware of the fact that
completly transparent VMs is a double-edged sword.m

%%% Local Variables:
%%% mode: latex
%%% TeX-master: "../paper"
%%% End:
