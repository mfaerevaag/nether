\mvf{maybe a sentence or two, explaining how our paper is closely related to
  ether and nether, and the structure of this section}

\subsection{Ether}
Ether~\cite{ether} is a malware analysis platform that utilizes the Intel VT-x
hardware virtualization extensions, and theoretically has no presence in the
guest operating system. It uses native CPU instructions, thus does not suffer
from incomplete or inaccurate system emulation such as hardware emulators do.
Ether also comes with a rich feature set, thus it can monitor all the memory
write attempts of the guest, trace the instructions and system calls of in-guest
processes, and unpack a wide range of protected binaries. As a result, malware
cannot detect the presence of Ether. In that paper, they evaluated Ether and
several other state-of-the-art analyzers on the obfuscation techniques used to
obfuscate 25,000 recent malware samples. The results show that Ether remains
transparent and defeats the obfuscation tools that evade the existing
approaches. Ether's system architecture is represented in Figure~\ref{fig:ether}
below.

\begin{figure}[!h]
	\centering
	\includegraphics[width=\linewidth]{figure/ether.png}
	\caption{Ether's system architecture.}
	\label{fig:ether}
\end{figure}

\subsection{nEther}
P{\'e}k \textit{et al.} introduced novel approaches that make the detection of
hardware assisted virtualization platforms and out-of-the-guest malware analysis
frameworks possible~\cite{nether}. In order to demonstrate their concepts, they
implemented a scalable and flexible application framework over Window XP called
\textit{nEther} which can practically disclose the presence of both
\textit{Ether}~\cite{ether}, and Intel VT-x by executing multiple feature tests.
The system overview of nEther is depicted in Figure~\ref{fig:nether} and their
proposed feature tests consist of three parts as follows: 1) timing information,
2) CPUID information, and 3) CPU Errata.

The use of timing information is a well-known techinique to detect the presence
of traditional debuggers. The simplest way to get timing information is the use
of internal timers and periodic interrupt sources such as the Time-Stamp
Counter(TSC), Periodic Interrupt Timer (PIT), and so on. In case of
out-of-the-guest malware analyzer, it falsify internal timers in order to fool a
timing test and provide transparency. However, the correct manipulation of
timings cannot be guaranteed in all circumstances. As a result, nEther can
detect the existence of Ether, based on a contradiction raised by clock cycle
manipulation where the analyzer returns an adjusted cycle difference to hide its
presence.

The CPUID instruction returns processor identification and feature information 
depending on the content of general registers EAX and ECX. These could be exploited 
by anlyzer detectors as various modifications of the system can be retrieved. 
In case of Ether, it does alter a few bits of information that could be returned 
by the correct configuration of CPUID. Therefore, nEther can detect Ether 
by checking a few bits returned by a CPUID that are specific to Ether.

CPU Errata refer to the collection of design defects or errors in specific model
or family. nEther exploits bugs in the Core 2 Duo family, called the AH4 Erratum
which unintendedly updates the Last Exception Record (LER) MSR. This errata is
not implemented in hardware-assisted virtualization solutions, such as Xen. In
other words, this bug will not happen in the hardware-assisted virtualized
runtime environment, and as a result, nEther can distinguish virtualized
environment from native by executing CPU errata and checking whether LER MSR is
unexpectedly updated or not. These type of feature tests could not reveal the
presence of Ether, but could reveal the hardware-assisted virtualized runtime
environment, such as Xen.

\begin{figure}[!h]
	\centering
	\includegraphics[width=\linewidth]{figure/nether.png}
	\caption{nEther's system overview.}
	\label{fig:nether}
\end{figure}

\subsection{Detecting Virtualized Environments}
Raffetseder \textit{et al.} proposed approaches to identify a processor using
CPU errata~\cite{raffetseder2007} and it is also valid for detecting system
monitors. They exploit \textit{Errata N 5} and \textit{Errata N 86} of the Intel
Pentium 4 processor, that result in a segment fault or an incorrect debug
exception respectively. In QEMU, however, both errata does either not occur or
the effects are different from native environment. As a result, they can detect
system monitors, but this is only valid for the QEMU hardware emulator.

Several efforts has been made on how relative difference in execution time of
two instructions could be used to detect a virtualized
environments~\cite{raffetseder2007, thompson}. Using absolute time differences
is not feasible, as architectures are so are complex and different that absolute
timings are not practical. Relative differences, on the other hand, is more
predictable. Comparing one instruction that is known not to be trapped by the
VMM and one that is, for instance {\tt NOP} and {\tt CPUID}, the authors showed
that different VMMs showed significantly different ratios compared to bare
metal.

Ferrie \textit{et al.}~\cite{ferrie2007} showed in 2007 how context switches
between a VMM and guest could be used to detect hypervisors based on Intel VT-x
through the flushing of the Translation Lookaside Buffer (TLB). Using a
non-privileged instruction that is still trapped by the VMM, i.e. {\tt CPUID},
will cause a flush of the TLB. This could be detected through timing the
instruction before and after anticipated flush. This method has later been
documented by other authors~\cite{thompson}.

Franklin \textit{et al.} proposed an approach to detect VMM by exploiting the
timing dependency exception property of a virtual machine
monitor~\cite{franklin2008}. Their approach can detect several popular VMMs,
including VMware and Xen.

There are several attempts to defeat detection of virtualized environments. Kang
\textit{et al.} proposed an approach to dynamically modify the execution of a
whole-system emulator in order to fool a malware and thwart anti-virtualization
techniques using in malware~\cite{kang2009}. Their approach uses a scalable
trace matching algorithm to locate the point where the emulated execution
diverges. From this point, the states of the reference system and the emulator
are compared to create a dynamic state modification that can repair the
difference.

%%% Local Variables:
%%% mode: latex
%%% TeX-master: "../paper"
%%% End:
